\documentclass{article}
\usepackage{amsmath}
\usepackage{geometry}

\begin{document}

\section*{My Formula Sheet}

\subsection*{Calculus Formulas}
\begin{align}
   lim_{x \to \frac{3\pi}{4}} \frac{sin(x)}{cos(x)}\\
   lim_{x \to \frac{3\pi}{4}} tan(x)\\
   lim_{x \to \frac{3\pi}{4}} tan(\frac{3\pi}{4}) = -1
\end{align}
\[\lim_{x \to 0} \ln{ \lvert x \rvert} = -\infty \]
\[\frac{1}{\cos{x}} = \sec{x}\]
\[\lim_{x \to 0} \lvert \frac{1}{x} \rvert = +\infty \] 
\[\ln{x}=n; x = e^{n}\]
\[\ln{e^x}=x\]
\[\frac{1}{\lvert \frac{1}{x} \rvert} = \lvert x \rvert\]
\[\lim_{x \to 0}\frac{1}{(x-1)^2}=+\infty\]
\[\lim_{x \to 0}\frac{1}{x}=+\infty\]
\[\frac{d}{dx} \ln{x} = \frac{1}{x}\]
\[\ln{1} = 0\]
\[\frac{d}{dx} \sec{x} = \sec{x} \tan{x}\]
\[\frac{d}{dx} e^x = e^x\]
\[\frac{d}{dx} e^{-x} = -e^{-x}\]
\[\frac{d}{dx} \tan{x} = \sec^2{x}\]
\[\frac{d}{dx} a^x = a^x \ln{a}\]
\[\sin{0} = 0\]
\[\cos{0} = 1\]
\[\tan{0} = 0\]
\[\sec{0} = 1\]
\[\cot{0} = undefined\]
\[\csc{0} = undefined\]
\[\ln{x}, x<0 = undefined\]
\[\lim_{x \to 0^+} \ln{x} = -\infty\]
\[\lim_{x \to 0^+} \frac{1}{x} = \infty\]
\[\lim_{x \to 0^+} x \ln{x} = \lim_{x \to 0^+} \frac{\ln{x}}{\frac{1}{x}} = \lim_{x \to 0^+} \frac{\frac{1}{x}}{\frac{-1}{x^2}} = \lim_{x \to 0^+} -x = 0\]
Product rule: \(f(x)g(x) = f(x)g'(x)+f'(x)g(x)\)\\
\newline
Chain Rule: \(F(x)=f'(g(x))g(x)\)\\
\newline
Quotient rule: \(\frac{f(x)}{g(x)} = \frac{g(x)f'(x)-f(x)g'(x)}{(g(x)^2)}\)\\
\newline
The quadratic formula is given by, use it when you \underline{cannot factor}, for example:
\[-141 = -9x^2+2x-2\]
Here, we cannot solve for x, so we put \(-141\) to the right and use that \(ax^2+bx+c\):
\[0 = -9x^2+2x+139\]
\[ x = \frac{-b \pm \sqrt{b^2 - 4ac}}{2a} \]
\[ x = \frac{-(2) \pm \sqrt{(2)^2 - 4(-9)(139)}}{2(-9)} \]
\[ x = \frac{-(2) \pm \sqrt{5008}}{-18} \]
\[ x = \frac{(2) \pm \sqrt{5008}}{18} \]
Now here, we can reduce that square root by \((\sqrt{16} \sqrt{313})\), here \((16)(313) = 5008\), and we know that \(\sqrt{16} = 4\):
\[ x = \frac{(2) \pm \sqrt{16} \sqrt{313}}{18} \]
\[ x = \frac{(2) \pm 4 \sqrt{313}}{18} \]
Therefore, we end up with:
\[ x = \frac{(2) + 4 \sqrt{313}}{18} = 4.04262356\]
and
\[ x = \frac{(2) - 4 \sqrt{313}}{18} = -3.82040134\]
\textbf{Difference of Squares}:
\[a^2-b^2 = (a+b)(a-b)\]
\textbf{Perfect Square Trinomial}: 
\[a^2 \pm 2ab + b^2 = (a+b)^2\]






\end{document}
