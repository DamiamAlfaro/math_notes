\documentclass[12pt, letterpaper]{article}
\usepackage{amsmath}
\usepackage{graphicx} % Required for inserting images
\newcommand{\ltf}[1]{\lim_{{#1} \to \infty}}
\newcommand{\nltf}[1]{\lim_{{#1} \to -\infty}}
\title{14 Limits of Ratios of Functions; Indeterminate Forms; LHopital’s Rule}
\author{Damiam Alfaro}
\date{December 3rd 2023}

\begin{document}
\maketitle

\section{First Part}
\textbf{Goal}: Understanding when and why the limit of a ratio of two functions at a point is determinate or indeterminate.\\
\newline
\textbf{Motivation}: Many applications involve ratios of functions whose limit one needs to determine.\\
\newline
Limits of ratios of functions for when \(\lim_{x \to a}f(x)=L\) and \(\lim_{x \to a}g(x)=M\), letting L and M be finite or infinite, then
\[\lim_{x \to a}f(x)=L\]
\[\lim_{x \to a}g(x)=M\]
\[\lim_{x \to a}\frac{f(x)}{g(x)} = \frac{\lim_{x \to a}f(x)}{\lim_{x \to a}g(x)}\]
\begin{enumerate}
    \item \(\frac{L}{M}\), if L and M are finite and M \(\neq 0\)
    \item \(+\infty\), if L \(= +\infty\) and \(M > 0\) or \(L=-\infty\) and \(M <0\)
    \item \(-\infty\), if L \(= -\infty\) and \(M > 0\) or \(L=+\infty\) and \(M <0\)
    \item 0, if L is finite and \(M = +\infty\) or \(M=-\infty\)
    \item All these are determinate forms
\end{enumerate}
Take the case of \((b)\) for example:
\[\lim_{x \to a}f(x)=L\]
\[\lim_{x \to a}g(x)=M\]
b. \(+\infty\) if L \(= +\infty\) and \(M > 0\) or \(L=-\infty\) and \(M <0\)\\
\newline
Take for example:
\[f(x) = \frac{1}{(x-1)^2}\]
\[g(x) = sinx\]
\[a = 1\]
Then
\[\lim_{x \to a}\frac{1}{(x-1)^2}=+\infty\]
Why is the result of the equation above \(\infty\)? Because we consider reciprocals of very small positive numbers, taking the reciprocal of a number approaching 0 results in a value that goes to positive infinity.\\
\newline
\[\lim_{x \to a}\sin{x}=\sin{1} >0\]
\[\lim_{x \to 1}\frac{f(x)}{g(x)}=\lim_{x \to 1}\frac{\frac{1}{(x-1)^2}}{\sin{x}}=\lim_{x \to 1}\frac{1}{\sin{x}(x-1)^2}=+\infty\]
Which concludes in \((b)\)
\(+\infty\) if L \(= +\infty\) and \(M > 0\) or \(L=-\infty\) and \(M <0\)
\[\frac{+\infty}{\sin{1}}\]

\section{Second Part}
\textbf{Goal}: Understanding that when the limit of a ratio of two functions at a point is indeterminate, it may or may not exist.\\
\newline
\textbf{Motivation}: Many applications involve ratios of functions whose limit may or may not exist.\\
\newline
There are other several other possibilities for limits of ratios of functions, the ratio of limits exists in a generalized sense for determinate forms, but it may or may not exist for indeterminate forms. Again, when \(\lim_{x \to a}f(x)=L\) and \(\lim_{x \to a}g(x)=M\), letting L and M be finite or infinite, then:
\[\lim_{x \to a}f(x)=L\]
\[\lim_{x \to a}g(x)=M\]
\[\lim_{x \to a}\frac{f(x)}{g(x)} = \frac{\lim_{x \to a}f(x)}{\lim_{x \to a}g(x)}\]
\begin{enumerate}
    \item \(L = M = 0\)
    \item \(L = \pm \infty\) and \(M = \pm \infty\)
    \item \(L \neq 0\) and \(M=0\)
\end{enumerate}
These are called indeterminate forms, in all three cases the limit may or may not exist\\
\newline
The first two, \(\frac{0}{0}\) and \(\frac{\infty}{\infty}\) fall into L'Hopital's Rule\\
\newline
In the third case, the \(\lim_{x \to a}\frac{f(x)}{g(x)}\) does not exist unless \(g(x) \neq 0\) for all \(x\) near \(a\). For example:

\section{Third Part - L'Hôpital's Rule}
\textbf{Goal}: Understanding when and why the limit of a ratio of two functions at a point is equal to the limit of the ratio of their derivatives.
\textbf{Motivation}: Some applications involve ratios of functions that may both approach zero or infinity at the same time, or other indeterminate forms.\\
\newline
When it comes to indeterminate forms of type \(\frac{0}{0}\) and \(\frac{\infty}{\infty}\), the theorem applies if:
\begin{enumerate}
    \item \(f\) and \(g\) differentiable for all inputs \(x\) near \(a\)
    \item \(\lim_{x \to a} f(x)\) and \(lim_{x \to a} g(x)\) exist and are both \(\infty\) or both \(0\)
    \item \(\lim_{x \to a}\frac{f'(x)}{g'(x)}\) exist, then \(\lim_{x \to a}\frac{f(x)}{g(x)} = \lim_{x \to a}\frac{f'(x)}{g'(x)}\) 
\end{enumerate}
\(f'\) and \(g'\) have to be continuous near given number and the limit of \(\frac{f'}{g'}\) has to exist.\\
\newline
\textit{To be Continued...}








 
\end{document}
