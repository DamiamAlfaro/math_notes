\documentclass[12pt, letterpaper]{article}
\usepackage{amsmath}
\usepackage{graphicx} % Required for inserting images
\newcommand{\ltf}[1]{\lim_{{#1} \to \infty}}
\newcommand{\nltf}[1]{\lim_{{#1} \to -\infty}}
\title{14 Limits of Ratios of Functions; Indeterminate Forms; LHopital’s Rule}
\author{Damiam Alfaro}
\date{December 3rd 2023}

\begin{document}
\maketitle

\section{First Part}
\textbf{Goal}: Understanding when and why the limit of a ratio of two functions at a point is determinate or indeterminate.\\
\newline
\textbf{Motivation}: Many applications involve ratios of functions whose limit one needs to determine.\\
\newline
Limits of ratios of functions for when \(\lim_{x \to a}f(x)=L\) and \(\lim_{x \to a}g(x)=M\), letting L and M be finite or infinite, then
\[\lim_{x \to a}f(x)=L\]
\[\lim_{x \to a}g(x)=M\]
\[\lim_{x \to a}\frac{f(x)}{g(x)} = \frac{\lim_{x \to a}f(x)}{\lim_{x \to a}g(x)}\]
\begin{enumerate}
    \item \(\frac{L}{M}\), if L and M are finite and M \(\neq 0\)
    \item \(+\infty\), if L \(= +\infty\) and \(M > 0\) or \(L=-\infty\) and \(M <0\)
    \item \(-\infty\), if L \(= -\infty\) and \(M > 0\) or \(L=+\infty\) and \(M <0\)
    \item 0, if L is finite and \(M = +\infty\) or \(M=-\infty\)
    \item All these are determinate forms
\end{enumerate}
Take the case of \((b)\) for example:
\[\lim_{x \to a}f(x)=L\]
\[\lim_{x \to a}g(x)=M\]
b. \(+\infty\) if L \(= +\infty\) and \(M > 0\) or \(L=-\infty\) and \(M <0\)\\
\newline
Take for example:
\[f(x) = \frac{1}{(x-1)^2}\]
\[g(x) = sinx\]
\[a = 1\]
Then
\[\lim_{x \to a}\frac{1}{(x-1)^2}=+\infty\]
Why is the result of the equation above \(\infty\)? Because we consider reciprocals of very small positive numbers, taking the reciprocal of a number approaching 0 results in a value that goes to positive infinity.\\
\newline
\[\lim_{x \to a}\sin{x}=\sin{1} >0\]
\[\lim_{x \to 1}\frac{f(x)}{g(x)}=\lim_{x \to 1}\frac{\frac{1}{(x-1)^2}}{\sin{x}}=\lim_{x \to 1}\frac{1}{\sin{x}(x-1)^2}=+\infty\]
Which concludes in \((b)\)
\(+\infty\) if L \(= +\infty\) and \(M > 0\) or \(L=-\infty\) and \(M <0\)
\[\frac{+\infty}{\sin{1}}\]

\section{Second Part}
\textbf{Goal}: Understanding that when the limit of a ratio of two functions at a point is indeterminate, it may or may not exist.\\
\newline
\textbf{Motivation}: Many applications involve ratios of functions whose limit may or may not exist.\\
\newline
There are other several other possibilities for limits of ratios of functions, the ratio of limits exists in a generalized sense for determinate forms, but it may or may not exist for indeterminate forms. Again, when \(\lim_{x \to a}f(x)=L\) and \(\lim_{x \to a}g(x)=M\), letting L and M be finite or infinite, then:
\[\lim_{x \to a}f(x)=L\]
\[\lim_{x \to a}g(x)=M\]
\[\lim_{x \to a}\frac{f(x)}{g(x)} = \frac{\lim_{x \to a}f(x)}{\lim_{x \to a}g(x)}\]
\begin{enumerate}
    \item \(L = M = 0\)
    \item \(L = \pm \infty\) and \(M = \pm \infty\)
    \item \(L \neq 0\) and \(M=0\)
\end{enumerate}
These are called indeterminate forms, in all three cases the limit may or may not exist\\
\newline
The first two, \(\frac{0}{0}\) and \(\frac{\infty}{\infty}\) fall into L'Hopital's Rule\\
\newline
In the third case, the \(\lim_{x \to a}\frac{f(x)}{g(x)}\) does not exist unless \(g(x) \neq 0\) for all \(x\) near \(a\). For example:

\section{Third Part - L'Hôpital's Rule}
\textbf{Goal}: Understanding when and why the limit of a ratio of two functions at a point is equal to the limit of the ratio of their derivatives.
\textbf{Motivation}: Some applications involve ratios of functions that may both approach zero or infinity at the same time, or other indeterminate forms.\\
\newline
When it comes to indeterminate forms of type \(\frac{0}{0}\) and \(\frac{\infty}{\infty}\), the theorem applies if:
\begin{enumerate}
    \item \(f\) and \(g\) differentiable for all inputs \(x\) near \(a\)
    \item \(\lim_{x \to a} f(x)\) and \(lim_{x \to a} g(x)\) exist and are both \(\infty\) or both \(0\)
    \item \(\lim_{x \to a}\frac{f'(x)}{g'(x)}\) exist, then \(\lim_{x \to a}\frac{f(x)}{g(x)} = \lim_{x \to a}\frac{f'(x)}{g'(x)}\) 
\end{enumerate}
\(f'\) and \(g'\) have to be continuous near given number and the limit of \(\frac{f'}{g'}\) has to exist and one has to determine the indeterminate type, i.e. \(\frac{0}{0}\) or \(\frac{\infty}{\infty}\) by inputting \(a\) in the functions. What L'Hôpital's Rule does is that it disregards those forms and gives you a real solution instead: 
\[\lim_{x \to 3}\frac{\ln{x-2}}{x-3}\]
First input \(a\) to determine the indeterminate form:
\[\lim_{x \to 3}\frac{\ln{(3)-2}}{(3)-3}\]
\[\lim_{x \to 3}\frac{0}{0}\]
Now, here's where L'Hôpital's Rule applies, by using the derivatives of both functions \(\lim_{x \to a}\frac{f(x)}{g(x)} = \lim_{x \to a}\frac{f'(x)}{g'(x)}\)
\[f(x) = \ln{(3)-2}\]
\[f'(x) = \frac{1}{x-2}(1)\]
Here we used chain rule
\[g(x) = x-3\]
\[g'(x) = 1\]
Therefore we have:
\[\lim_{x \to 3}\frac{f'(x)}{g'(x)} = \frac{\frac{1}{x-2}}{1}\]
\[\lim_{x \to 3}\frac{f'(x)}{g'(x)} = \frac{1}{x-2}\]
Now if we approach the limit again, we get:
\[\lim_{x \to 3}\frac{f'(x)}{g'(x)} = \frac{1}{3-2} = 1\]
You have to keep in mind that the result might either be \(\pm \infty\) or \(0\) still, but in extreme cases, the result might be either \(\frac{0}{0}\) or \(\frac{\infty}{\infty}\) again, in those cases one has to get the double, triple, or n-th derivative until we are left with constants, see example:
\[\lim_{x \to 0} \frac{\sec{x}-\cos{x}}{x^2}\]
Here we see that if we input \(a\) we see that it gives us \(\frac{0}{0}\), therefore we do L'Hôpital's Rule
\[\lim_{x \to 0}\frac{f'(x)}{g'(x)} = \frac{\sec{x}\tan{x}+\sin{x}}{2x}\]
However, if we input \(a\) one more time, we will still get \(\frac{0}{0}\), we will solve until we are left with constants, therefore we do L'Hôpital's Rule once again:
\[\lim_{x \to 0} \frac{\sec{x}\tan^2{x}+\sec^3{x}+\cos{x}}{2}\]
Now, if we input \(a\) we will get:
\[\lim_{x \to 0} \frac{\sec{0}\tan^2{0}+\sec^3{0}+\cos{0}}{2}\]
\[\lim_{x \to 0} \frac{0+1+1}{2} = 1\]

\section{Fourth Part - L'Hôpital's Rule Part 2}
\textbf{Goal}: Understanding when an algebraic transformation may allow the use of L'Hôpital's Rule for an expression that is not a ratio.
\textbf{Motivation}: Some applications involve indeterminate forms that are not ratios.\\
\newline
We might have indeterminate products: \(\lim_{x \to a} f(x)g(x)\) if the limit of one of the functions is \(0\) and the other is \(\pm \infty\), then we have \((0) (\infty)\). Here we use some algebraic tricks to turn a \underline{product} into a \underline{quotient}:
\[f(x)g(x) = \frac{f(x)}{\frac{1}{g(x)}}\]
or also: 
\[f(x)g(x) = \frac{g(x)}{\frac{1}{f(x)}}\]
Here then we can use L'Hôpital's Rule since we will be having either \(\frac{0}{0}\) or \(\frac{\infty}{\infty}\). One of the scenarios may increase the complexity of the equation, this is where one tries both scenarios \(\frac{f(x)}{\frac{1}{g(x)}}\) and \(\frac{g(x)}{\frac{1}{f(x)}}\).\\
\newline
In the case of \underline{Indeterminate Differences} where \(\lim_{x \to a} f(x)-g(x)\) and their limits are \(\pm \infty\) we encounter an Indeterminate form type \(\infty - \infty\). Here we can turn it into a quotient by:
\[\lim_{x \to a} f^2(x)-g^2(x) = \lim_{x \to a} \frac{(f(x)-g(x))(f(x)+g(x))}{f(x)+g(x)}\]
\[\lim_{x \to a} f(x)-g(x) = \lim_{x \to a} \frac{(f(x)-g(x))(f(x)+g(x))}{f(x)+g(x)}\]
Therefore, yielding a type \(\frac{0}{0}\) or \(\frac{\infty}{\infty}\) where we can appy L'Hôpital's Rule to, for example:
\[\lim_{x \to +\infty} \sqrt{4x^2+2} - 2x+3\]
\[f(x)=\sqrt{4x^2+2}\]
\[g(x)=2x-3\]
The indeterminate form here is \underline{Inderterminate Differences} \(\infty-\infty\), therefore we apply the theorem for these cases:
\[\lim_{x \to a} f(x)-g(x) = \lim_{x \to a} \frac{(f(x)-g(x))(f(x)+g(x))}{f(x)+g(x)}\]
\[\lim_{x \to a} \frac{(f(x)-g(x))(f(x)+g(x))}{f(x)+g(x)} = \lim_{x \to \infty} \frac{(\sqrt{4x^2+2} - 2x+3)(\sqrt{4x^2+2} + 2x+3)}{\sqrt{4x^2+2} + 2x+3}\]
\[\lim_{x \to \infty} \frac{(4x^2+2 - (2x+3)^2)}{\sqrt{4x^2+2} + 2x+3}\]
\[\lim_{x \to \infty} \frac{4x^2+2 - (4x^2-12x+9)}{\sqrt{4x^2+2} + 2x+3}\]
\[\lim_{x \to \infty} \frac{12x-7}{\sqrt{4x^2+2} + 2x+3}\]
This product now gives us \(\frac{\infty}{\infty}\), therefore we use L'Hôpital's Rule, however in this case we can use some algebraic forms to dismantle the equation by multiplying by \(\frac{1}{x}\):
\[\lim_{x \to \infty} \frac{(12x-7)\frac{1}{x}}{(\sqrt{4x^2+2} + 2x+3)\frac{1}{x}}\]
Now, here the \(\frac{1}{x}\) will enter the square root as a \(\frac{1}{x^2}\) in order to be able to enter, only in the square root though.
\[\lim_{x \to \infty} \frac{12-\frac{7}{x}}{\sqrt{4+\frac{2}{x^2}} + 2x-\frac{3}{x}}\]
\[\lim_{x \to \infty} \frac{12}{\sqrt{4}+2} = 3\]
The last type is \underline{Indeterminate Powers} where \(\lim_{x \to a} f(x)^{g(x)}\) and if the limit of both functions is \(0\) or \(\lim_{x \to a} f(x) = 1\) and \(\lim_{x \to a} g(x) = \pm \infty\) or \(\lim_{x \to a} f(x) = +\infty\) and \(\lim_{x \to a} g(x) = 0\) which yields the following indeterminate forms of exponential type: \(0^0\); \(1^\infty\); \(\infty^0\)\\
\newline
In order to obtain a new function is by taking the \(\ln{x}\):
\[h(x) = \ln{f(x)^{g(x)}} = g(x)\ln{f(x)}\]
This turns the equation into a product and we can use the first method in this part by turning the product into a quotient and then apply L'Hôpital's Rule. Once we find \(h(x)\), we need to put it as an exponent of \(e\), i.e. \(e^{h(x)}\)

"We divide men into three basic types, according to whether their motive is; knowledge, success, or gain."
-Plato
 
\end{document}
